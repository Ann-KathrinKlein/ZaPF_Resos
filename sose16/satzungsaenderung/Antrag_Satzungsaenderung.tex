\documentclass[draft,10pt,oneside]{scrartcl}

% Sprache und Encodings
\usepackage[ngerman]{babel}
\usepackage[T1]{fontenc}
\usepackage[utf8]{inputenc}

% Typographisch interessante Pakete
\usepackage{microtype} % Randkorrektur und andere Anpassungen

% References to Internet and within the document
\usepackage[pdftex,colorlinks=false,
pdftitle={Antrag zur Änderung der Geschäftsordnung für Plenen der ZaPF},
pdfauthor={Jörg Behrmann (FUB)},
pdfcreator={pdflatex},
pdfdisplaydoctitle=true]{hyperref}

% Absaetze nicht Einruecken
\setlength{\parindent}{0pt}
\setlength{\parskip}{2pt}

% Formatierung auf A4 anpassen
\usepackage{geometry}
\geometry{paper=a4paper,left=15mm,right=15mm,top=10mm,bottom=10mm}

\begin{document}

\section*{Antrag zur Änderung der Satzung der ZaPF}

\textbf{Antragsteller:} Jörg Behrmann (FUB), Björn Guth (RWTH)

\subsection*{Antrag}

Hiermit beantragen wir die Satzung der ZaPF wie folgendt zu ändern:

In §5\,(d) ersetze
\begin{quote}
	Die ZaPF und jDPG entsenden je ein Mitglied in das Kommunikationsgremium.
\end{quote}
durch
\begin{quote}
	Die ZaPF entsendet zwei Mitglieder in das Kommunikationsgremium.
\end{quote}

Füge anschließend
\begin{quote}
	Davon beginnt die Amtszeit eines Mitgliedes auf einer ZaPF im Sommersemester und
	die des anderen Mitgliedes auf einer ZaPF im Wintersemester.
\end{quote}
ein.

\subsection*{Begründung}
Dies bildet den aktuellen Status quo ab. Außerdem können wir der jDPG nicht
vorschreiben, wie viele Mitglieder sie in das Kommunikationsgremium entsendet.

\newpage

\section*{Antrag zur Änderung der Satzung der ZaPF}

\textbf{Antragsteller:} Jörg Behrmann (FUB), Björn Guth (RWTH)

\subsection*{Antrag}

Hiermit beantragen wir die Satzung der ZaPF wie folgendt zu ändern:

In §5\,(b) ersetze
\begin{quote}
	Der StAPF besteht aus maximal fünf Physik-Studierenden (...)
\end{quote}
durch
\begin{quote}
	Der StAPF besteht aus maximal fünf natürlichen Personen (...)
\end{quote}

\subsection*{Begründung}
Durch die bisherige Formulierung war es bei strenger Auslegung der Satzung nicht
möglich Studierende, die einen anderen Studiengang als Physik studieren, in den
StAPF zu wählen. Dies schließt unter anderem Lehramtsstudierende, Studierende
aus Bindestrich-Studiengängen und alte Säcke aus.

Durch die neue Formulierung werden an dieser Stelle keine Menschen
ausgeschlossen. Wer genau gewählt werden kann, wird dann durch die Vergabe des
passiven Wahlrechts in Absatz 4.2.1 der Geschäftsordnung der ZaPF geregelt.

\newpage

\section*{Antrag zur Änderung der Satzung der ZaPF}

\textbf{Antragsteller:} Jörg Behrmann (FUB), Björn Guth (RWTH)

\subsection*{Antrag}

Hiermit beantragen wir die Satzung der ZaPF wie folgendt zu ändern:

In §5\,(b) nach
\begin{quote}
	Die Entscheidungen innerhalb des StAPF müssen in diesen Fällen einstimmig fallen.
\end{quote}
füge
\begin{quote}
	Der StAPF ist beschlussfähig falls mindestens drei seiner Mitglieder auf einer
	Sitzung anwesend sind und der Beschluss in der Sitzungseinladung angekündigt
	wurde.
\end{quote}
ein.

\subsection*{Begründung}
Bisher regelt die Satzung nicht eindeutig, wann genau der StAPF beschlussfähig
ist. Zum einen kann der Satz so ausgelegt werden, dass alle StAPF-Mitglieder
einstimmig einen Beschluss fassen müssen, zum anderen ist auch die Auslegung
gerechtfertigt, dass alle auf einer Sitzung anwesenden StAFPF-Mitglieder bei
Einstimmigkeit beschlussfähig sind. Dies wird durch die Ergänzung präzessiert.
Außerdem wird allen Interessierten die Möglichkeit gegeben, an der Debatte des
StAPFes teilzunehmen und so auf die Beschlussfassung einzuwirken.

\newpage

\section*{Antrag zur Änderung der Satzung der ZaPF}

\textbf{Antragsteller:} Jörg Behrmann (FUB), Björn Guth (RWTH)

\subsection*{Antrag}

Hiermit beantragen wir die Satzung der ZaPF wie folgendt zu ändern:

In §5\,(b) ersetze
\begin{quote}
	Sollte kein StAPF gewählt werden übernimmt das Plenum der ZaPF die Aufgaben
	des StAPF.
\end{quote}
durch
\begin{quote}
	Sollten alle Posten des StAPFes vakant sein, übernehmen die von der ZaPF
	entsandten Mitglieder des Kommunikationsgremiums oder, falls diese vakant sind,
	die Mitglieder des Technischen Organisationsausschuss aller Physikfachschaften
	oder, falls auch diese vakant sind, die Mitglieder der letzten die ZaPF
	ausrichtenden Fachschaft die Archivierungs- und Veröffentlichungsaufgaben des
	StAPF.
\end{quote}

\subsection*{Begründung}
Da die Aufgaben des StAPFes die ZaPF zwischen den ZaPFen vertritt, ergibt es
kenen Sinn, dass die Aufgaben des StAPFes bei Nichtwahl eines StAPFes durch das
Plenum der ZaPF übernommen werden. Außerdem ist dies auch durch die Natur der
Aufgaben schlicht nicht möglich. Daher soll dies in Zukunft Organen und
Organisationen, die durch ihre eigenen Aufgaben und die nähe zum StAPF eher
geeignet sind, diese Aufgaben zu übernehmen.

\end{document}
