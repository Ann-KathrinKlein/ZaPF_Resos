\documentclass[draft,10pt,oneside]{scrartcl}

% Sprache und Encodings
\usepackage[ngerman]{babel}
\usepackage[T1]{fontenc}
\usepackage[utf8]{inputenc}

% Typographisch interessante Pakete
\usepackage{microtype} % Randkorrektur und andere Anpassungen

% References to Internet and within the document
\usepackage[pdftex,colorlinks=false,
pdftitle={Antrag zur Änderung der Geschäftsordnung für Plenen der ZaPF},
pdfauthor={Jörg Behrmann (FUB), Björn Guth (RWTH)},
pdfcreator={pdflatex},
pdfdisplaydoctitle=true]{hyperref}

% Absaetze nicht Einruecken
\setlength{\parindent}{0pt}
\setlength{\parskip}{2pt}

% Formatierung auf A4 anpassen
\usepackage{geometry}
\geometry{paper=a4paper,left=15mm,right=15mm,top=10mm,bottom=10mm}

\begin{document}

\section*{Antrag zur Änderung der Geschäftsordnung für Plenen der ZaPF}

\textbf{Antragsteller:} Jörg Behrmann (FUB), Björn Guth (RWTH)

\subsection*{Antrag}

Hiermit beantragen wir die Geschäftsordnung für Plenen der ZaPF wie folgendt zu ändern:

In 3.2.4 ersetze den Punkt
\begin{quote}
	geheime Abstimung (ohne Gegenrede, ohne Abstimmung, setzt namentliche
	Abstimmung außer Kraft)
\end{quote}
durch
\begin{quote}
	geheime Abstimung (ohne Gegenrede, ohne Abstimmung, setzt namentliche
	Abstimmung und Abstimmung per Handzeichen außer Kraft)
\end{quote}
sowie den Punkt
\begin{quote}
	namentliche Abstimmung (ohne Gegenrede, ohne Abstimmung)
\end{quote}
durch
\begin{quote}
	namentliche Abstimmung (ohne Gegenrede, ohne Abstimmung, setzt Abstimmung
	per Handzeichen außer Kraft)
\end{quote}
Außerdem füge den Punkt
\begin{quote}
	Abstimmung per Handzeichen (ohne Gegenrede, ohne Abstimmung, nur bei
	Abstimmungen und Meinungsbildern)
\end{quote}
\vspace{0.25cm}
In 4.1.5 ersetze
\begin{quote}
	Die Abstimmung geschieht durch deutliches Handheben, eine geheime
	Abstimmung kann beantragt werden.
\end{quote}
durch
\begin{quote}
	Die Abstimmung ist geeignet, z.B. durch deutliches Handheben, kenntlich zu
	machen, eine geheime Abstimmung in Papierform kann beantragt werden.
\end{quote}

\subsection*{Begründung}
Diese Änderung erlaubt die Nutzung anderer Wahlmethoden als Handzeichen, wie
z.B. die Clicker in den Konstanzer Plenen.

\newpage

\section*{Antrag zur Änderung der Geschäftsordnung für Plenen der ZaPF}

\textbf{Antragsteller:} Jörg Behrmann (FUB), Björn Guth (RWTH)

\subsection*{Antrag}

Hiermit beantragen wir die Geschäftsordnung für Plenen der ZaPF wie folgendt zu ändern:

In 4.2.1 ersetze
\begin{quote}
	Das passive Wahlrecht für Personenwahlen haben alle angemeldeten Personen
\end{quote}
durch
\begin{quote}
	Das passive Wahlrecht für Personenwahlen haben alle teilnehmenden Personen
\end{quote}

\subsection*{Begründung}
Durch die alte Formulierung besitzen Helfika und Organisorika kein passives
Wahlrecht und können nicht in Funktionen der ZaPF gewählt werden, da sie keine
Teilnehmika der ZaPF sind. Der Bergriff \glqq{}teilnehmede Personen\grqq{} wird
in 1 genauer definiert.

\newpage

\section*{Antrag zur Änderung der Geschäftsordnung für Plenen der ZaPF}

\textbf{Antragsteller:} Jörg Behrmann (FUB), Björn Guth (RWTH)

\subsection*{Antrag}

Hiermit beantragen wir die Geschäftsordnung für Plenen der ZaPF wie folgendt zu ändern:

In 2.7 füge
\begin{quote}
	Auf einer vorherigen ZaPF vertagten Anträge sind priorisiert zu behandeln.
\end{quote}
als letztes ein.

\subsection*{Begründung}
Diese Einfügung soll der Praxis, sich mit Anträge durch eine Vertagung de facto
nicht zu befassen, bzw. der Gefahr Anträge aufgrund einer späten Platzierung auf
der Tagesordnung und einer daraus folgenden Beschlussunfähigkeit des Plenums vor
einer möglichen Abstimmung über mehrere ZaPFen vorbeugen.

\newpage

\section*{Antrag zur Änderung der Geschäftsordnung für Plenen der ZaPF}

\textbf{Antragsteller:} Jörg Behrmann (FUB), Björn Guth (RWTH)

\subsection*{Antrag}

Hiermit beantragen wir die Geschäftsordnung für Plenen der ZaPF wie folgendt zu ändern:

In 2.2 füge
\begin{quote}
	Bis zur Wahl der Sitzungsleitung fungiert die ausrichtende Fachschaft als
	Sitzungsleitung.
\end{quote}
als letztes ein.
\vspace{0.25cm}

Weiter füge in 4.2.2
\begin{quote}
	In Abweichung davon dürfen Sitzungsleitung und Protokollführung per
	Akklamation gewählt werden.
\end{quote}
als letztes ein

\subsection*{Begründung}
Bisher ist nirgends geregelt, durch wen die Sitzungsleitung kommissarisch
ausgeübt wird bis eine Sitzungsleitung gewählt wurde. Dies wird durch die erste
Einfügung nun getan.

Weiter muss die Sitzungsleitung und die Protokollführung nach aktueller
Geschäftsordnung gemäß den Regeln zu Personenwahlen in 4.2 geheim gewählt
werden. Da dies nicht der real gelebten Praxis entspricht, gemäß der beide per
Akklamation gewählt werden. Dies ist mit der zweiten Einfügung dann auch formal
möglich.

\newpage

\section*{Antrag zur Änderung der Geschäftsordnung für Plenen der ZaPF}

\textbf{Antragsteller:} Jörg Behrmann (FUB), Björn Guth (RWTH)

\subsection*{Antrag}

Hiermit beantragen wir die Geschäftsordnung für Plenen der ZaPF wie folgendt zu ändern:

In 4.2.4 streiche
\begin{quote}
	Eine geheime Abstimmung ist möglich.
\end{quote}

\subsection*{Begründung}
Dies ist eine Dopplung, da in 4.2.2 schon definiert ist, dass Personenwahlen immer geheim durchzuführen sind.

\end{document}
