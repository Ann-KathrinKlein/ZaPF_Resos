\documentclass[draft,12pt,oneside]{scrartcl}

% Sprache und Encodings
\usepackage[ngerman]{babel}
\usepackage[T1]{fontenc}
\usepackage[utf8]{inputenc}

% Typographisch interessante Pakete
\usepackage{microtype} % Randkorrektur und andere Anpassungen

% References to Internet and within the document
\usepackage[pdftex,colorlinks=false,
pdftitle={Resolution zu Studentischen Beschäftigungsverhältnissen nach dem WissZeitVG},
pdfauthor={Jörg Behrmann (Freie Universität Berlin)},
pdfcreator={pdflatex},
pdfdisplaydoctitle=true]{hyperref}

% Absaetze nicht Einruecken
\setlength{\parindent}{0pt}
\setlength{\parskip}{2pt}

% Formatierung auf A4 anpassen
\usepackage{geometry}
\geometry{paper=a4paper,left=20mm,right=20mm,top=10mm,bottom=10mm}

\begin{document}

\section*{Resolution zu Studentischen Beschäftigungsverhältnissen nach dem WissZeitVG}

\textbf{Antragsteller:} Jörg Behrmann (Freie Universität Berlin)

\textbf{Empfänger:} Hochschulrektorenkonferenz, deutschsprachige Hochschulen,
Kultusministerkonferenz, Bundesministerium für Bildung und Forschung

\subsection*{Resolution zu Studentischen Beschäftigungsverhältnissen nach dem WissZeitVG}

Die ZaPF empfiehlt, dass die Regelung des \textsection{}6
Wissenschaftszeitvertragsgesetz so ausgelegt werden soll, dass alle
studentischen Anstellungsverhältnisse an Hochschulen als künstlerische oder
wissenschaftliche Hilfstätigkeiten anzusehen sind.

\subsection*{Begründung}

Die Begrenzung der Regelung des WissZeitVG auf wissenschaftliche und
künstlerische Hilfstätigkeiten führt dazu, dass manche Universitäten bestimmte
studentische Hilfskraftstellen, z.B. in der Univerwaltung, konservativ nicht als
solche Hilfstätigkeiten auslegen und sie aus diesem Grund nach dem Teilzeit- und
Befristungsgesetz sachgrundlos befristen. Dies hat zur Folge, dass diese
Verträge nur für zwei Jahre abgeschlossen werden und nicht verlängert werden
können. Dies lehnen wir ab.

\end{document}

%%% Local Variables:
%%% mode: latex
%%% TeX-master: t
%%% End:
