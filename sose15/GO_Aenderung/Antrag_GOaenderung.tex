\documentclass[draft,12pt,oneside]{scrartcl}

% Sprache und Encodings
\usepackage[ngerman]{babel}
\usepackage[T1]{fontenc}
\usepackage[utf8]{inputenc}

% Typographisch interessante Pakete
\usepackage{microtype} % Randkorrektur und andere Anpassungen

% References to Internet and within the document
\usepackage[pdftex,colorlinks=false,
pdftitle={Antrag zur Änderung der Geschäftsordnung für Plenen der ZaPF},
pdfauthor={Jörg Behrmann (FUB)},
pdfcreator={pdflatex},
pdfdisplaydoctitle=true]{hyperref}

% Absaetze nicht Einruecken
\setlength{\parindent}{0pt}
\setlength{\parskip}{2pt}

% Formatierung auf A4 anpassen
\usepackage{geometry}
\geometry{paper=a4paper,left=20mm,right=20mm,top=10mm,bottom=10mm}

\begin{document}

\section*{Antrag zur Änderung der Geschäftsordnung für Plenen der ZaPF}

\textbf{Antragsteller:} Jörg Behrmann (FUB)

\subsection*{Antrag}

Hiermit beantragen wir den Anhang zu Geschäftsordnungsanträgen wie folgt zu ändern.

\begin{quote}
Geschäftsordnungsanträge sind dazu gedacht, zu verhindern, dass eine Diskussion
sich ins Absurde zieht. Sie sind mit äußerster Vorsicht anzuwenden und sind
insbesondere als Korrektiv für eine Diskussion, die ihren roten Faden verloren
hat, zu benutzen.

Bei der Abstimmung über einen Geschäftsordnungsantrag sollte man vorher dreimal
darüber nachdenken, ob man ihm zustimmt, da Ende der Debatte auch Ende der Debatte
bedeutet.

Geschäftsordnungsanträge können als Mittel zu einer Schlammschlacht genutzt
werden, jedoch sollte bedacht werden, dass wir sachliche Diskussionen führen
wollen und auch einsehen sollten, wenn die Mehrheit einen Antrag nicht
unterstützt. Die GO kann nie so gefasst werden, dass sie weder von Teilnehmenden
des Plenums noch von der Redeleitung missbraucht werden kann. Für einen guten
Ablauf des Plenums sind wir auf das Wohlwollen aller angewiesen.

Um die GO-Anträge auf ihren einzigen Sinn, die Steuerung der Diskussion, zu
beschränken, wurden auf der ZaPF im Wintersemester 2014/2015 in Bremen die Liste
der GO-Anträge abgeschlossen und umfasst alle GO-Anträge die in der jüngeren
Vergangenheit benutzt wurden und die, die schon immer auf der Liste waren.
Dies umfasst unter anderem auch Verfahrensvorschläge,
wie z.B. die Entscheidung 2011 in Dresden eine ZaPF, um die sich mehrere
Fachschaften beworben hatten, per Stein-Schere-Papier zu vergeben.

Falls ein GO-Antrag nicht wie in der Liste benannt gestellt wird, versucht die
Redeleitung in Rücksprache einen inhaltsgleichen, korrekt gestellten Antrag zu
finden. Sollte die Redeleitung dabei einen Fehler macht, erinnert euch daran,
dass auch die Redeleitung nur aus Menschen besteht, die Fehler machen können und
weist sie darauf hin.

Abstimmungen ohne jegliche Gegenrede sollten nur mit äußerster Vorsicht
angenommen werden.

Formale Gegenrede bedeutet nur bekanntzugeben, dass man dagegen ist, inhaltliche
Gegenrede beinhaltet eine Begründung.
\end{quote}

\subsection*{Begründung}

Dieser Kommentar soll die Überlegungen hinter der Schließung der Liste der
GO-Anträge auf der letzten ZaPF dokumentieren.

\end{document}
