\documentclass[draft,12pt,oneside]{scrartcl}

% Sprache und Encodings
\usepackage[ngerman]{babel}
\usepackage[T1]{fontenc}
\usepackage[utf8]{inputenc}

% Typographisch interessante Pakete
\usepackage{microtype} % Randkorrektur und andere Anpassungen

% References to Internet and within the document
\usepackage[pdftex,colorlinks=false,
pdftitle={Antrag zur Änderung der Geschäftsordnung für Plenen der ZaPF},
pdfauthor={Jörg Behrmann (FUB)},
pdfcreator={pdflatex},
pdfdisplaydoctitle=true]{hyperref}

% Absaetze nicht Einruecken
\setlength{\parindent}{0pt}
\setlength{\parskip}{2pt}

% Formatierung auf A4 anpassen
\usepackage{geometry}
\geometry{paper=a4paper,left=20mm,right=20mm,top=10mm,bottom=10mm}

\begin{document}

\section*{Antrag zur Änderung der Satzung der ZaPF}

\textbf{Antragsteller:} Jörg Behrmann (FUB)

\subsection*{Antrag}

Hiermit beantragen wir der Satzung der ZaPF folgendes Gremium hinzuzufügen.

\begin{quote}
\textbf{Der Technische Organisationsausschuss aller Physikfachschaften (TOPF)}

Der Technische Organisationsausschuss aller Physikfachschaften (TOPF) ist für
die Instandhaltung und Dokumentation der EDV-Projekte der ZaPF verantwortlich.

Er besteht aus zwei vom Plenum zu bestimmenden Personen, die für die
Aufrechterhaltung des Betriebs und die Dokumentation der Basissysteme
hauptverantwortlich sind, und einer beliebigen Anzahl von freiwilligen Helfern,
die für die Dokumentation und den Betrieb von einzelnen Projekten verantwortlich
sind.

Die Hauptverantwortlichen sind dem Plenum und dem StAPF rechenschaftspflichtig
und an ihre Weisungen gebunden. Insbesondere hat das Plenum die Möglichkeit,
Datenschutzerklärungen und Nutzungsordnungen sowohl für das Gesamtsystem als
auch für einzelne Projekte zu bestimmen.

Die freiwilligen Helfer werden nicht gewählt, sondern durch die beiden
Hauptverantwortlichen gemeinsam bestimmt. Sie sind ihnen rechenschaftspflichtig
sowie an deren Weisungen und die erlassenen Ordnungen gebunden.

Die Wahl der beiden Hauptverantwortlichen ist eine Personenwahl entsprechend der
Geschäftsordnung für Plenen der ZaPF.  Je ein Hauptverantwortlicher wird zur
ZaPF im Winter- und der andere im Sommersemester gewählt.

\end{quote}

\subsection*{Begründung}

Die ZaPF hat über die Jahre viel Infrastruktur angesammelt, die leider oftmals
nur unzureichend dokumentiert ist. Oftmals ist es Glückssache Leute zu finden,
die diese Dinge am laufen halten und manchmal geht Wissen um ihren Betrieb
verloren, da die Betreiber nicht mehr zu ZaPFen fahen.

Auch ist der StAPF nicht das geeignete Gremium um diese Arbeit zu machen, obwohl
er sie in der Vergangenheit stellenweise getan hat, da er genug andere Dinge zu
betreuen hat. Aus diesem Grund wird dieses neue Gremium vogrschlagen.

\end{document}
