\documentclass[12pt,oneside]{scrartcl}

% Sprache und Encodings
\usepackage[ngerman]{babel}
\usepackage[T1]{fontenc}
\usepackage[utf8]{inputenc}
\usepackage{epigraph}

% Typographisch interessante Pakete
\usepackage{microtype} % Randkorrektur und andere Anpassungen

% References to Internet and within the document
%\usepackage[pdftex,colorlinks=false,
%pdftitle={Resolution für mehr Lebensqualität},
%pdfauthor={Björn Guth (Aachen), Jörg Behrmann (FUB), Wolfgang Bauer (Alter Sack) und der Rest des Git-Workshops},
%pdfcreator={pdflatex},
%pdfdisplaydoctitle=true]{hyperref}

% Absaetze nicht Einruecken
\setlength{\parindent}{0pt}
\setlength{\parskip}{2pt}

% Formatierung auf A4 anpassen
\usepackage{geometry}
\geometry{paper=a4paper,left=20mm,right=20mm,top=10mm,bottom=10mm}

\renewcommand*\dictumwidth{0.95\linewidth}

\hyphenation{anonymer ano-ny-mer}
\begin{document}

\section*{Resolution zur gleichen Qualität von  eduroam und anderen
  hochschulöffentlichen Netzwerken an allen Hochschulen}

\textbf{Adressaten:} An alle Fachschaften der unterzeichnenden BuFaTas der ZKK
in Aachen, den DFN-Verein und die GÉANT Association

\subsection*{Antrag:}
Die teilnehmenden BuFaTas der ZKK in Aachen mögen beschließen:
\begin{quote}
  Die zeichnenden Bundesfachschaftentagungen begrüßen das weit verbreitete
  Angebot von eduroam an deutschsprachigen Hochschulen und halten die
  Qualitätssicherung des eduroam-Netzwerkes für die universitäre Arbeit für
  unerlässlich.

  Wir halten folgende Punkte für besonders kritisch und möchten daher auf diese
  explizit hinweisen.

  \begin{enumerate}
  \item Wir fordern die Einhaltung der eduroam Policy Service Definition,
    festgelegt von der GÉANT Association, in der Version 2.8 vom Juli 2012, da
    in der Vergangenheit von einigen Hochschulen einige Empfehlungen sowie
    Forderungen hierin nicht beachtet wurden.

    Herausheben wollen wir dabei
    \begin{enumerate}
    \item  die Einhaltung der in Abschnitt 6.3.3, Unterpunkt ``Network'',
      aufgeführten Liste der unbedingt anzubietenden Ports. Leider wurden wir
      auf zahlreiche Verstöße gegen diesen Punkt aufmerksam gemacht.

      Wir unterstützen darüber hinaus die Empfehlung keine
      bzw. möglichst wenige Portrestriktionen vorzunehmen, sowie keine
      Anwedungs- und Abfangproxies zu verwenden.

    \item die Einhaltung der in Abschnitt 6.3.2 festgelegten Unterstützung von
      anonymer Authentifizierung. Wir bitten diese Unterstützung auch in den
      entsprechenden Anleitungen zu dokumentieren.
    \end{enumerate}
  \item Falls Portrestriktionen unumgänglich sind, sollten diese öffentlich
    zugänglich dokumentiert und begründet werden, sowohl für ein- als auch für
    ausgehende Beschränkungen.

    Wir bitten die GÉANT Association dies in die eduroam Policy Service Definition
    als ``MUST''-Requirement aufzunehmen.

  \item Aufgrund der herausragenden Bedeutung des eduroam Netzes für die
    wissenschaftliche Gemeinschaft fordern wir eine ausreichende Ausstattung mit
    personellen und finanziellen Mitteln zur Aufrechterhaltung, zur Verbesserung
    und zum Ausbau des Netzwerkes.
  \end{enumerate}

  Wir bitten diese Hinweise analog auch für andere hochschulöffentliche Netze zu
  beherzigen.

\end{quote}

\subsection*{Erläuterung:}

\vspace{1cm}
\textbf{Verfasser:} Björn Guth (RWTH), Jörg Behrmann (FUB), Fabian Freyer (TUB),
Friedrich Zahn (TU Dresden), Sebastian Schrader (TU Dresden), Dennis Baurichter (Uni Paderborn)
und viele weitere.

\end{document}
