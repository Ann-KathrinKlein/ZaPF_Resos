\documentclass[draft,12pt,oneside]{scrartcl}

% Sprache und Encodings
\usepackage[ngerman]{babel}
\usepackage[T1]{fontenc}
\usepackage[utf8]{inputenc}

% Typographisch interessante Pakete
\usepackage{microtype} % Randkorrektur und andere Anpassungen

% References to Internet and within the document
\usepackage[pdftex,colorlinks=false,
pdftitle={Antrag zur Änderung der Geschäftsordnung für Plenen der ZaPF},
pdfauthor={Jörg Behrmann (FUB), Benjamin Dummer (HUB), Zafer El-Mokdad (Potsdam), Björn Guth (RWTH Aachen)},
pdfcreator={pdflatex},
pdfdisplaydoctitle=true]{hyperref}

% Absaetze nicht Einruecken
\setlength{\parindent}{0pt}
\setlength{\parskip}{2pt}

% Formatierung auf A4 anpassen
\usepackage{geometry}
\geometry{paper=a4paper,left=20mm,right=20mm,top=10mm,bottom=10mm}

\begin{document}

\section*{Antrag zur Änderung der Geschäftsordnung für Plenen der ZaPF}

\textbf{Antragsteller:} Jörg Behrmann (FUB), Benjamin Dummer (HUB), Björn Guth (RWTH Aachen), Zafer El-Mokdad (Potsdam)

\subsection*{Antrag}

Hiermit beantragen wir die vorliegende Entwurfsfassung für die Geschäftsordnung für Plenen der ZaPF als neue
Geschäftsordnung für Plenen der ZaPF zu bestätigen.

\subsection*{Begründung}

Die derzeitige Fassung der Geschäftsordnung für Plenen der ZaPF ist aufgrund der Ergebnisse des AK Anti-Harassment auf der Winter-ZaPF in Wien
2013 nicht mehr aktuell. Die dort beschlossenen Vertrauenspersonen bedürfen eines eigenen Wahlverfahrens, dass
in der Geschäftsordnung verankert und erläutert werden muss.

Darüber hinaus nutzen wir diesen Anlass um weitere notwendige Änderungen in die Geschäftsordnung einzuarbeiten. Diese sind
zweierlei:
\begin{enumerate}
\item Das Hinzufügen eines Geltungsbereiches und Einführen von Erläuterungen und Definitionen ungeklärter Begriffe, da
      diese derzeit fehlen. Diese Begriffe umfassen unter anderem den Wahlausschuss, wer eine angmeldete Person ist,
      sowie wer das passive Wahlrecht genießt.
      Dies sind inhaltliche Änderungen.
\item Das Aufspalten und umsortieren von Paragraphen und Absätzen in thematisch abgeschlossenen Einheiten.
      Dies sind redaktionelle Änderungen.
\item Das Präzisieren der Regelung zur Stimmabgabe in Abwesenheit, da ansonsten Unklarheit darüber herrscht
      wie die Stimmabgabe nach einem Änderungsantrag zu werten ist. Dies ist eine inhaltliche Änderung.
\item Das Anfügen und Erläutern des Wahlmodus der Vertrauenspersonen. Dies ist eine inhaltliche Änderung.
\item Das Auslagern von Fußnoten und Kommentaren in einen Kommentaranhang. Dies ist eine redaktionelle Ändrung.
\item Das Anfügen der Änderungshistorie in einem Anhang.
\item Im Wunsch eine geschlechtergerechte Sprache zu verwenden wird das generische Maskulinum durch Beidnennungen
      sowie neutrale Formulierungen ersetzt. Dies ist eine redaktionelle Änderung.
\end{enumerate}

Alle redaktionellen Änderungen, mit Ausnahme der Verwendunge geschlechtergerechterer Sprache,
dienen der Verbesserung der Lesbarkeit der Geschäftsordnung.

\subsection*{Die Änderungen im Überblick}

Die inhaltlichen Änderungen sind
\begin{enumerate}
\item der neu hinzugekommene Paragraph (1) aufgrund dessen sich alle anderen Paragraphennummern
      um eins erhöhen,
\item Definition einer angemeldeten Person im Geltungsbereich, Paragraph (1),
\item Das Einfügen der Vertrauenspersonenwahl im Anfangsplenum in Paragraph (2),
\item Die Definition von Beschlüssen und Meinungsbilder, Resolutionen, Positionspapieren,
      normalen Personenwahlen, Vertrauenspesonenwahlen in Paragraph (4) und die Definition des Wahlausschusses, sowie
\item Die neu hinzugekommen Absätze 6 bis 8 in Paragraph (4.2), welche die Vertrauenspersonenwahl
      regeln.
\end{enumerate}

\end{document}
