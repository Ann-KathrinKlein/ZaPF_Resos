\documentclass[draft,12pt,oneside]{scrartcl}

% Sprache und Encodings
\usepackage[ngerman]{babel}
\usepackage[T1]{fontenc}
\usepackage[utf8]{inputenc}

% Typographisch interessante Pakete
\usepackage{microtype} % Randkorrektur und andere Anpassungen

% References to Internet and within the document !!!always last package!!!
\usepackage[pdftex,colorlinks=false,
pdftitle={Antrag zur Änderung der Satzung der ZaPF},
pdfauthor={Jörg Behrmann (FUB), Benjamin Dummer (HUB), Björn Guth (RWTH Aachen), Zafer El-Mokdad (Potsdam)},
pdfcreator={pdflatex},
pdfdisplaydoctitle=true]{hyperref}

% Absaetze nicht Einruecken
\setlength{\parindent}{0pt}
\setlength{\parskip}{2pt}


\begin{document}

\section*{Antrag zur Änderung der Satzung der ZaPF}

\textbf{Antragsteller:} Jörg Behrmann (FUB), Benjamin Dummer (HUB), Björn Guth (RWTH Aachen), Zafer El-Mokdad (Potsda)

\subsection*{Antrag}

Hiermit beantragen wir die vorliegende Entwurfsfassung für die Satzung der ZaPF als neue Satzung der ZaPF zu bestätigen.

\subsection*{Begründung}

Die derzeitige Fassung der Satzung ist aufgrund der Ergebnisse des AK ``Anti-Harassment'' auf der Winter-ZaPF in Wien
2013 nicht mehr aktuell. Die dort beschlossenen Vertrauenspersonen müssen als Organ der ZaPF in die Satzung eingefügt
werden um die Notwendigkeit ihrer in der Geschäftsordnung für Plenen der ZaPF zu regelnden Wahl zu legitimieren.

Darüber hinaus nutzen wir diesen Anlass um weitere notwendige Änderungen in die Satzung einzuarbeiten. Diese sind
zweierlei
\begin{enumerate}
\item Das Kommunikationsgremium ist ein erfolgreiches de-facto-Organ der ZaPF, das seit mehrere Jahren gewählt wird.
      Die heutige Satzungsänderung fügt auch das Kommunikationsgremium in die Satzung ein. Dies ist eine inhaltliche
      Änderung.
\item Der letzte Satz der Aufgaben der ZaPF ``Eine ZaPF beginnt mit dem Anfangsplenum und endet nach dem Abschlussplenum.''
      wird in den Paragraphen (4) Tagung verschoben, da dies den Inhalt besser erfasst. Dies ist eine inhatliche Änderung.
\item Um die Lesbarkeit der Satzung zu erhöhen wurde der Text umformatiert und eine fehlerhafte Doppelnennung entfernt.
      Dies sind redaktionelle Änderungen.
\item Im Wunsch eine geschlechtergerechte Sprache zu verwenden wird das generische Maskulinum durch Beidnennungen
      sowie neutrale Formulierungen ersetzt. Dies ist eine redaktionelle Änderung.
\item Das Anfügen der bisherigen Änderungshistorie als Anhang.
\end{enumerate}

\subsection*{Die Änderungen im Überblick}

Die inhaltlichen Änderungen sind
\begin{enumerate}
\item das Verschieben des letzten Satzes von Paragraph (3) in Paragraph (4),
\item der neu hinzugekommene Paragraph (5) Absatz (c), sowie
\item der neu hinzugekommene Paragraph (5) Absatz (d).
\end{enumerate}

Die redaktionellen Änderungen sind
\begin{enumerate}
\item das Umwandeln aller männlichen Formulierungen in Beidnennung von männlicher und weiblicher Form,
\item das Einfügen von Absätzen im gesamten Text, sowie
\item das Löschen des gedoppelten sechsten Satzes in Paragraph (5) Absatz (b) über die Zusammensetzung des StAPF,
      was beinahe wortgleich schon im zweiten Satz des selben Absatzes geregelt wird.
\end{enumerate}

\end{document}
