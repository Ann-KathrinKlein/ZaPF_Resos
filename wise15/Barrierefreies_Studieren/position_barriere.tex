Die ZaPF empfiehlt den einzelnen Fachschaften:
\begin{itemize}
\item Eine Liste der Beratungsstellen zusammenzutragen und diese in geeigneter
	Form allgemein zugänglich zu machen. 
\item Bei den Beratungsstellen anfragen, ob sie Informationen zum
	Barrierefreien Studieren, dem an der Universität üblichen
	Informationsmaterial für Studierende im ersten Semester (z.\,B.
	Erstireader, Erstitüten, Mappen o.\,ä.), beifügen.
\item Die Beratungsstellen darauf hinzuweisen, ebenfalls darüber zu
	informieren, dass auch nicht offentsichtliche Gruppen (z.\,B. chronisch
	kranke Menschen, Menschen mit Teilleistungsstörungen oder sonstiger
	Studienbeeinträchtigung), Nachteilsausgleiche erhalten dürfen.
\item Bei dem Behindertenbeauftragten anfragen, ob ein Informationsvortrag zu
	dieser Thematik Eingang in die Erstiveranstaltungen finden kann.
\end{itemize}
