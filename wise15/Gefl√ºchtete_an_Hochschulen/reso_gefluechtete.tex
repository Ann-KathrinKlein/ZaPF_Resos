Die ZaPF fordert einen freien Zugang zu allgemeinen Bildungsressourcen und -
infrastrukturen (öffentl. Bibliotheken, Volkshochschulen, Goethe-Institut etc.)
für Geflüchtete. Insbesondere soll für Geflüchtete die Aufnahme bzw.
Weiterführung eines Studiums vereinfacht werden. 

Es sollte für Geflüchtete, die aktuell keine  Hochschulzugangsberechtigung oder
Nachweise über bisherige Studienleistungen vorlegen können, unabhängig von
ihrem Aufenthaltsstatus die Möglichkeit bestehen, Studienleistungen zu
erbringen\footnote{d.h. vorbehaltlich einer späteren Prüfung der
	nachzureichenden Hochschulzugangsberechtigung bzw. Nachweise über bisherige
Studienleistungen Prüfungen ablegen zu dürfen}.

Darüber hinaus soll der Zugang zur universitären Infrastruktur gegeben sein
(z.B. Hochschulsport, Hochschulnetzwerke, Computerpool, Sprachkurse). Der Bund
und die Länder haben dafür Sorge zu tragen, dass die genannten Maßnahmen weder
den Geflüchteten noch den Hochschulen zu Lasten fallen. Neben der
Studienfinanzierung über das BAföG muss es verstärkt unkomplizierte
Stipendienprogramme geben, um Geflüchtete gezielt zu unterstützen. Die
Hochschulen haben als Orte gesellschaftlichen Fortschrittes eine Vorbild- und
Vorreiterfunktion, insbesondere sollten sie mittels eines offenen und
respektvollen Umgangs mit Geflüchteten öffentlich Stellung beziehen, um
Vorurteilen und Fremdenfeindlichkeit entgegen zu treten. Zudem muss
sichergestellt werden, dass die Informationen über die verschiedenen Angebote  %öffentlichkeitswirksam 
zielgruppengerecht in transparenter und kompakter Art und Weise zur Verfügung
stehen.
