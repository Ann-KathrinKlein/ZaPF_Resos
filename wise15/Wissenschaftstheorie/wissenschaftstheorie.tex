\subsection*{Ziel}
Die Zusammenkunft aller Physikfachschaften (ZaPF) fordert eine stärkere Einbindung wissenschaftstheoretischer und ethischer Fragestellungen im Physikstudium. Hierbei soll vor allem eine Sensibilisierung für diese Themen bei den Studierenden sowie eine Verbesserung wissenschaftlicher Qualifikationsarbeiten erreicht werden. In welcher Art und Weise die Themen in das Curriculum integriert werden, soll von den lokalen Gegebenheiten der einzelnen Hochschulen abhängen.

\subsection*{Verpflichtende Themenfelder im Studium}
Die ZaPF spricht sich dafür aus, dass die Physikstudierenden folgende Themenfelder verpflichtend in ihrem Studium eingehend behandeln, damit sich hierbei eine selbstkritische Betrachtung des eigenen Handelns entwickelt. Eine solche Entwicklung verbessert nicht nur die Qualität der wissenschaftlichen Arbeiten, etwa der Qualifikationsarbeiten. Auch der Erkenntnisgewinn durch wissenschaftliche Arbeiten kann so erhöht werden.
\begin{itemize}
\item\textbf{Was ist gute wissenschaftliche Praxis?}
    \par Es ist unabdingbar, dass Studierende im Studienverlauf die Grundsätze guter wissenschaftlicher Praxis in der Physik  erlernen. Daher unterstützt die ZaPF die Bestrebungen der KFP \footnote{Konferenz der Fachbereiche Physik, \url{http://www.kfp-physik.de}}, eine  Handreichung zu diesem Thema zu erarbeiten. Die ZaPF strebt darüber  hinaus die konsequente Einbindung dieser Inhalte in das Curriculum an. Dabei ist zu beachten, dass diese noch vor der Anfertigung der  Qualifikationsarbeiten erfolgen muss. Wichtige Themen sind hierbei etwa der korrekte Umgang mit wissenschaftlicher Literatur und geistigem Eigentum anderer, die Eigenständigkeit und Originität der Arbeit, der korrekte Umgang mit experimentellen und numerisch erhaltenen Daten, die Aufbewahrung der Primärdaten, die transparente Darstellung und Offenlegung von externen Einflüssen sowie die Dokumentation von Nullergebnissen.
\item\textbf{Wie funktioniert wissenschaftliche Theoriebildung?}
    \par Es ist unserer Ansicht nach für Studierende wichtig, ein Grundverständnis wissenschaftstheoretischer Überlegungen zur Theoriebildung zu besitzen. Hierzu gehören etwa Konzepte der Verifikation und Falsifikation sowie Überlegungen zu Grenzen der wissenschaftlichen Theoriebildung, etwa der Theoriebeladenheit der Beobachtung.
\item\textbf{Wie muss mit Messdaten umgegangen werden?}
    \par Kenntnisse des korrekten Umgangs mit Messdaten sind Grundvoraussetzung für jede Art von experimentellem wissenschaftlichen Arbeiten. Den Studierenden sollte daher beigebracht werden, wie sie mit Messabweichungen und Ähnlichem umzugehen haben.
    \end{itemize}

\subsection*{Weitergehende Themen im Studium}
Neben den oben genannten Themen, welche die ZaPF aufgrund ihrer großen Bedeutung für unverzichtbar halten, sollen folgende Themen den Studierenden im Wahlpflichtbereich oder als freiwillige Lehrveranstaltungen angeboten werden, damit sie sich hiermit weitergehend beschäftigen können:
\begin{itemize}
\item \textbf{Welche gesellschaftliche Verantwortung tragen Wissenschaftler*innen?}
    \par In allen Bereichen der Wissenschaft ist ein Bewusstsein für die Folgen des eigenen Forschungshandelns sehr wichtig. Die Studierenden sollten diesbezüglich zur kritischen Reflexion und zur eigenen ethischen Stellungnahme befähigt werden. Je nach Spezialisierung gehören hierzu etwa Fragen der Technikfolgenabschätzung oder der Bioethik.
\item \textbf{Was macht moderne Wissenschaft aus?}
    \par Die Interpretation der Resultate moderner Wissenschaft ist wie zuvor angedeutet stark anhängig von den zu Grunde gelegten Annahmen aus Erkenntnistheorie und Ontologie. Dabei spielt insbesondere die Frage \glqq Was ist Wissen?\grqq\ eine wichtige Rolle, da gerade in der Physik oftmals sehr komplexe Experimentier- und Messapparaturen, Simulationen und theoretische Konzepte verwendet werden. 
   \end{itemize}

\subsection*{Umsetzungsvorschläge}
\par Da die deutsche Hochschullandschaft sehr heterogen ist, ist eine einheitliche Einbindung der genannten Punkte in das Studium nicht sinnvoll. Dennoch sollen im Folgenden einige Beispiele genannt werden, wie die geforderten Punkte in das Studium eingebracht werden könnten.
Alle Physikstudierenden durchlaufen im Studium mindestens ein Laborpraktikum. Wir sehen dies als eine gute Gelegenheit, gute wissenschaftliche Praxis in das Physikstudium einzubringen. Sei es durch ein begleitendes Kolloquium oder durch eine verbesserte Schulung der Tutoren.
Auch eine Einbindung in andere Lehrveranstaltungen ist denkbar sowie das Anbieten einer Schlüsselqualifikation oder eines (Block-)Seminars. Auch eine Kooperation mit Philosophie-Fachbereichen oder Bibliotheken ist wünschenswert.\\
Wichtig ist bei all diesen Maßnahmen, dass die Studierenden nicht nur punktuell, sondern kontinuierlich über das gesamte Studium angesprochen werden.
