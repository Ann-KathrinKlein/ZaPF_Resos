\documentclass[draft,10pt,oneside]{scrartcl}

% Sprache und Encodings
\usepackage[ngerman]{babel}
\usepackage[T1]{fontenc}
\usepackage[utf8]{inputenc}

% Typographisch interessante Pakete
\usepackage{microtype} % Randkorrektur und andere Anpassungen

% References to Internet and within the document
\usepackage[pdftex,colorlinks=false,
pdftitle={Antrag zur Änderung der Geschäftsordnung für Plenen der ZaPF},
pdfauthor={Jörg Behrmann (FUB)},
pdfcreator={pdflatex},
pdfdisplaydoctitle=true]{hyperref}

% Absaetze nicht Einruecken
\setlength{\parindent}{0pt}
\setlength{\parskip}{2pt}

% Formatierung auf A4 anpassen
\usepackage{geometry}
\geometry{paper=a4paper,left=20mm,right=20mm,top=10mm,bottom=10mm}

\begin{document}

\section*{Antrag zur Änderung der Satzung der ZaPF}

\textbf{Antragsteller:} Jörg Behrmann (FUB), Björn Guth (RWTH)

\subsection*{Antrag}

Hiermit beantragen wir die Satzung der ZaPF wie folgendt zu ändern.

In §5 ersetze
\begin{quote}
	Die Organe der ZaPF sind das ZaPF-Plenum, der Ständige Ausschuss der
	Physik-Fachschaften (StAPF), die Vertrauenspersonen, das Kommunikationsgremium
	und der Technische Organisationsausschuss aller Physikfachschaften (TOPF).
\end{quote}
durch
\begin{quote}
	Die Organe der ZaPF sind das ZaPF-Plenum, der Ständige Ausschuss der
	Physik-Fachschaften (StAPF), die Vertrauenspersonen, das Kommunikationsgremium
	(KomGrem) und der Technische Organisationsausschuss aller Physikfachschaften
	(TOPF).
\end{quote}

Füge anschließend
\begin{quote}
	Die Wahlen von Mitgliedern des StAPF, des KomGrem und des TOPF sind
	Personenwahlen entsprechend der Geschäftsordnung der ZaPF.

	Die Mitgliedschaft im StAPF, dem Kommunikationsgremium oder dem TOPF endet, so
	nicht anders geregelt, mit Ablauf der Amtszeit, Ableben des Amtsinhabers oder
	der Amtsinhaberin, Niederlegung des Amtes oder Abwahl mit Zweiteldrittelmehrheit
	durch das Plenum. Der Antrag auf Abwahl ist bis 15:00 Uhr am Vortag bei der
	ausrichtenden Fachschaft anzukündigen.

	Bis zur Nachwahl bleibt ein unbesetztes Amt vakant. Bei der Nachwahl wird das
	Amt bis zum Ablauf der Restdauer der Amtszeit besetzt.
	Die Nachwahl findet zum nächstmöglichen Zeitpunkt auf einem Abschlussplenum
	einer Tagung statt.
	Sollten nach einer Wahl Posten unbesetzt sein, bleiben sie vakant.

	Falls mindestens zwei Drittel der Mitglieder eines Gremiums das Amt niederlegen
	endet auch die Amtszeit der übrigen Mitglieder.
\end{quote}
ein.
\vspace{0.25cm}

In §5(a) ersetze
\begin{quote}
	Zu jeder im Sommersemester stattfindenden ZaPF werden drei Mitglieder des StAPF
	neu gewählt.
	Zu jeder im Wintersemester stattfindenden ZaPF werden zwei Mitglieder des StAPF
	neu gewählt.

	Sollten ein oder mehrere Posten im StAPF vakant sein, muss im Abschlussplenum der
	darauf folgenden ZaPF eine Nachbesetzung durchgeführt werden.
	Die nachbesetzte Person bleibt für die Restdauer der Wahlperiode des
	ausgeschiedenen Mitgliedes im Amt.
	Die Nachbesetzung ist eine Personenwahl wie zur Wahl des gesamten StAPF.
	Sollte es keine Kandidatinnen oder Kandidaten für diese Posten geben, bleiben
	sie vakant.
\end{quote}
durch
\begin{quote}
	Die Amtszeit von drei Mitgliedern des StAPF beginnt zu einer im Sommersemester
	stattfindenden ZaPF und die zweier StAPF-Mitglieder zu einer im Wintersemester
	stattfindenden ZaPF.
\end{quote}
\vspace{0.25cm}

In §5(e) ersetze
\begin{quote}
	Die Wahl der beiden Hauptverantwortlichen ist eine Personenwahl entsprechend der
	Geschäftsordnung für Plenen der ZaPF.  Je ein Hauptverantwortlicher wird zur
	ZaPF im Winter- und der andere im Sommersemester gewählt.
\end{quote}
durch
\begin{quote}
	Die Amtszeit eines Hauptverantwortlichen beginnt zu einer im Sommersemester
	stattfindenden ZaPF, die des anderen zu einer im Wintersemester stattfindenden
	ZaPF.
\end{quote}

\subsection*{Begründung}

Bisher ist das Ausscheiden aus gewählten Gremien der ZaPF nicht klar geregelt.
Des weiteren werden aktuell vorhandene und durch die neuen Passagen entstandene
Redundanzen in den nachfolgenden Unterpunkten des §5 entfernt.

\end{document}
