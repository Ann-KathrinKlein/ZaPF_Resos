\section{Resolution zum Bildungszugang für Geflüchtete}
\subsection*{Stellungnahme}
Die ZaPF fordert einen freien Zugang zu allgemeinen Bildungsressourcen und -
infrastrukturen (öffentl. Bibliotheken, Volkshochschulen, Goethe-Institut etc.)
für Geflüchtete. Insbesondere soll für Geflüchtete die Aufnahme bzw.
Weiterführung eines Studiums vereinfacht werden. 

Es sollte für Geflüchtete, die aktuell keine  Hochschulzugangsberechtigung oder
Nachweise über bisherige Studienleistungen vorlegen können, unabhängig von
ihrem Aufenthaltsstatus die Möglichkeit bestehen, Studienleistungen zu
erbringen\footnote{d.h. vorbehaltlich einer späteren Prüfung der
	nachzureichenden Hochschulzugangsberechtigung bzw. Nachweise über bisherige
Studienleistungen Prüfungen ablegen zu dürfen}.

Darüber hinaus soll der Zugang zur universitären Infrastruktur gegeben sein
(z.B. Hochschulsport, Hochschulnetzwerke, Computerpool, Sprachkurse). Der Bund
und die Länder haben dafür Sorge zu tragen, dass die genannten Maßnahmen weder
den Geflüchteten noch den Hochschulen zu Lasten fallen. Neben der
Studienfinanzierung über das BAföG muss es verstärkt unkomplizierte
Stipendienprogramme geben, um Geflüchtete gezielt zu unterstützen. Die
Hochschulen haben als Orte gesellschaftlichen Fortschrittes eine Vorbild- und
Vorreiterfunktion, insbesondere sollten sie mittels eines offenen und
respektvollen Umgangs mit Geflüchteten öffentlich Stellung beziehen, um
Vorurteilen und Fremdenfeindlichkeit entgegen zu treten. Zudem muss
sichergestellt werden, dass die Informationen über die verschiedenen Angebote  %öffentlichkeitswirksam 
zielgruppengerecht in transparenter und kompakter Art und Weise zur Verfügung
stehen.

\subsection*{ZaPF intern}
Die Menschen, die aus Kriegsgebieten wie Syrien und dem Irak nach Deutschland
flüchten, haben zu einem großen Anteil eine Hochschulzugangsberechtigung in
ihrem Heimatland erlangt und es ist im Sinne des Bundes und der Länder, diese
Menschen weiter auszubilden. Kriegsvertriebene haben häufig das Problem
fehlender Dokumente, die eine Hochschulzugangsberechtigung oder
Studienleistungen belegen.

Daher ist es einerseits notwendig, Hilfestellung dabei zu leisten die fehlenden
Unterlagen nachzureichen und andererseits sollten diese schwierigen Umstände
den Studieneinstieg bzw. die Weiterführung eines Studiums nicht unnötig
verzögern.

Das deutsche Hochschulwesen ist als einziges in der Lage, schnell Geflüchtete
mit Internetzugang, durch Systeme wie eduroam und die schon vorhandenen
PC-Pools, zu versorgen. Dadurch wird der Zugang zu Informationen  und
Weiterbildungsmaßnahmen (z.B. Online-Sprachkurse, Programm der Kiron University
Berlin) sowie gesellschaftliche Teilhabe verbessert. Die schon jetzt
angespannte finanzielle Lage der Hochschulen darf durch die neuen
Herausforderungen (erhöhter Beratungsaufwand, höhere Studierendenzahlen etc.)
nicht weiter strapaziert werden, daher müssen zusätzliche Gelder aus Bund und
Ländern zu Verfügung gestellt werden.
%Adressaten: KFP, DPG,MeTaFa, HRK, KMK, LAKs
%Antragstellende: Timo (RWTH), BenniD (HUB)
