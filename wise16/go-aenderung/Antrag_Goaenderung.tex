\documentclass[draft,10pt,oneside]{scrartcl}

% Sprache und Encodings
\usepackage[ngerman]{babel}
\usepackage[T1]{fontenc}
\usepackage[utf8]{inputenc}

% Typographisch interessante Pakete
\usepackage{microtype} % Randkorrektur und andere Anpassungen

% References to Internet and within the document
\usepackage[pdftex,colorlinks=false,
pdftitle={Antrag zur Änderung der Geschäftsordnung für Plenen der ZaPF},
pdfauthor={Jörg Behrmann (FUB), Björn Guth (RWTH)},
pdfcreator={pdflatex},
pdfdisplaydoctitle=true]{hyperref}

% Absaetze nicht Einruecken
\setlength{\parindent}{0pt}
\setlength{\parskip}{2pt}

% Formatierung auf A4 anpassen
\usepackage{geometry}
\geometry{paper=a4paper,left=15mm,right=15mm,top=10mm,bottom=10mm}

\begin{document}

\section*{Antrag zur Änderung der Geschäftsordnung für Plenen der ZaPF}

\textbf{Antragsteller:} Jörg Behrmann (FUB), Björn Guth (RWTH)

\subsection*{Antrag}

Hiermit beantragen wir die Geschäftsordnung für Plenen der ZaPF wie folgendt zu
ändern:

In 4.1.3, 4.1.4 und 4.2.4 ersetze
\begin{quote}
	angemeldete Person
\end{quote}
durch
\begin{quote}
	teilnehmende Person
\end{quote}

\subsection*{Begründung}
Da die ausrichtende Fachschaft in der nicht zur ZaPF angemeldet sind, haben sie
nach aktueller Formulierung kein aktives Wahlrecht, da dies sowohl bei
Meinungsbildern als auch bei fachschaftenweiser Abstimmung durch die
Anwesenheit angemeldeter Personen definiert ist. Da die Gruppe der
teilnehmenden Personen gemäß 1 Abs. 2 der Geschäftsordnung für Plenen der ZaPF
auch explizit die Helferinnen und Helfer der ausrichtenden Fachschaft umfasst,
bekommen diese so auch das aktive Stimmrecht.

\newpage

\section*{Antrag zur Änderung der Geschäftsordnung für Plenen der ZaPF}

\textbf{Antragsteller:} Jörg Behrmann (FUB), Björn Guth (RWTH)

\subsection*{Antrag}

Hiermit beantragen wir die Geschäftsordnung für Plenen der ZaPF wie folgendt zu
ändern:

In 4.2.9 füge
\begin{quote}
	Der Antrag auf Abwahl ist bis spätestens 15 Uhr am Vortag der ausrichtenden
	Fachschaft anzukündigen.
\end{quote}
als zweiten Satz ein

\subsection*{Begründung}
Dies ist keine neue Regelung, sondern ein Passus, der bisher in der Satzung
geregelt wurde. Da es sich hierbei aber um das Verfahren der Abwahl handelt,
sollte dies in der Geschäftsordnung geregelt werden.

\end{document}
