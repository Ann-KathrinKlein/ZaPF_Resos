\documentclass[draft,10pt,oneside]{scrartcl}

% Sprache und Encodings
\usepackage[ngerman]{babel}
\usepackage[T1]{fontenc}
\usepackage[utf8]{inputenc}

% Typographisch interessante Pakete
\usepackage{microtype} % Randkorrektur und andere Anpassungen

% References to Internet and within the document
\usepackage[pdftex,colorlinks=false,
pdftitle={Antrag zur Änderung der Geschäftsordnung für Plenen der ZaPF},
pdfauthor={Jörg Behrmann (FUB), Björn Guth (RWTH)},
pdfcreator={pdflatex},
pdfdisplaydoctitle=true]{hyperref}

% Absaetze nicht Einruecken
\setlength{\parindent}{0pt}
\setlength{\parskip}{2pt}

% Formatierung auf A4 anpassen
\usepackage{geometry}
\geometry{paper=a4paper,left=15mm,right=15mm,top=10mm,bottom=10mm}

\begin{document}

\section*{Antrag zur Änderung der Geschäftsordnung für Plenen der ZaPF}

\textbf{Antragsteller:} Jörg Behrmann (FUB), Björn Guth (RWTH)

\subsection*{Antrag}

Hiermit beantragen wir die Geschäftsordnung für Plenen der ZaPF wie folgendt zu
ändern:

In 4.1.3, 4.1.4 und 4.2.4 ersetze
\begin{quote}
	angemeldete Person
\end{quote}
durch
\begin{quote}
	teilnehmende Person
\end{quote}

\subsection*{Begründung}
Da die ausrichtende Fachschaft in der nicht zur ZaPF angemeldet sind, haben sie
nach aktueller Formulierung kein aktives Wahlrecht, da dies sowohl bei
Meinungsbildern als auch bei fachschaftenweiser Abstimmung durch die
Anwesenheit angemeldeter Personen definiert ist. Da die Gruppe der
teilnehmenden Personen gemäß 1 Abs. 2 der Geschäftsordnung für Plenen der ZaPF
auch explizit die Helferinnen und Helfer der ausrichtenden Fachschaft umfasst,
bekommen diese so auch das aktive Stimmrecht.

\newpage

\section*{Antrag zur Änderung der Geschäftsordnung für Plenen der ZaPF}

\textbf{Antragsteller:} Jörg Behrmann (FUB), Björn Guth (RWTH)

\subsection*{Antrag}

Hiermit beantragen wir die Geschäftsordnung für Plenen der ZaPF wie folgendt zu
ändern:

In 3.2.4 ersetze
\begin{quote}
	zur Unterbrechung der Sitzung,
\end{quote}
durch
\begin{quote}
	zur Unterbrechung der Sitzung (auch bekannt als "Pause"),
\end{quote}
sowie
\begin{quote}
	zum Schluss der Debatte (die Diskussion wird nach Annahme des Antrages
	sofort abgebrochen, eine Abstimmung zum Thema wird ggf. sofort
	durchgeführt)*
\end{quote}
durch
\begin{quote}
	zum Schluss der Debatte (die Diskussion wird nach Annahme des Antrages
	sofort abgebrochen, eine Abstimmung zum Thema wird ggf. sofort
	durchgeführt, auch bekannt als "Antrag auf sofortige Abstimmung") *
\end{quote}
und
\begin{quote}
	zur Schließung der Redeliste und Verweisung in eine Arbeitsgruppe mit Recht
	auf ein Meinungsbild im Plenum *
\end{quote}
durch
\begin{quote}
	zur Schließung der Redeliste und Verweisung in eine Arbeitsgruppe mit Recht
	auf ein Meinungsbild im Plenum (auch bekannt als "Vertagung auf dienächste
	ZaPF") *
\end{quote}

\subsection*{Begründung}
Mit dieser Änderung versuchen wir der gehäuften Kritik entgegenzuwirken, dass
durch die geschlossene Liste der Geschäftsordnungsanträge und die recht
sperrigen Namen der der Anträge dazu führen, dass traditionell gestellte
Geschäftsordnungsanträge nicht mehr berücksichtigt werden können.

\newpage

\section*{Antrag zur Änderung der Geschäftsordnung für Plenen der ZaPF}

\textbf{Antragsteller:} Jörg Behrmann (FUB), Björn Guth (RWTH)

\subsection*{Antrag}

Hiermit beantragen wir die Geschäftsordnung für Plenen der ZaPF wie folgendt zu
ändern:

In 4 ersetze
\begin{quote}
	fünfzehn Physikfachschaften
\end{quote}
durch
\begin{quote}
	zwanzig Physikfachschaften
\end{quote}
sowie in 4.2.5 ersetze
\begin{quote}
	mindestens acht Ja-Stimmen
\end{quote}
durch
\begin{quote}
	mindestens elf Ja-Stimmen
\end{quote}
und ersetze im Anhang
\begin{quote}
	Beschlussfähigkeit bei fünfzehn anwesenden Fachschaften
\end{quote}
durch
\begin{quote}
	Beschlussfähigkeit bei zwanzig anwesenden Fachschaften
\end{quote}
sowie
\begin{quote}
	Das Minimum von acht Ja-Stimmen
\end{quote}
durch
\begin{quote}
	Das Minimum von elf Ja-Stimmen
\end{quote}

\subsection*{Begründung}
Die alte Regelung geht von 60 Physikfachschaften aus, die auf die ZaPF
eingeladen werden. In jüngerer Vergangenheit wurden aber regelmäßig ca. 80
Fachschaften zur ZaPF eingeladen. Die Erhöhung der Beschlussfähigkeit auf 20
Stimmen und die Mindestanzahl von Ja-Stimmen auf elf Stimmen passt diese
Regelungen an die aktuelle Situation an.

\newpage

\section*{Antrag zur Änderung der Geschäftsordnung für Plenen der ZaPF}

\textbf{Antragsteller:} Jörg Behrmann (FUB), Björn Guth (RWTH)

\subsection*{Antrag}

Hiermit beantragen wir die Geschäftsordnung für Plenen der ZaPF wie folgendt zu
ändern:

In 2.7 ersetze
\begin{quote}
	Auf einer vorherigen ZaPF durch einen GO-Antrag auf "Schließung der
	Redeliste und Verweisung in eine Arbeitsgruppe mit Recht auf ein
	Meinungsbild im Plenum" vertagten Anträge sollen priorisiert behandelt
	werden.
\end{quote}
durch
\begin{quote}
	Auf einer vorherigen ZaPF durch einen GO-Antrag auf "Schließung der
	Redeliste und Verweisung in eine Arbeitsgruppe mit Recht auf ein
	Meinungsbild im Plenum" vertagte Anträge sowie solche, die wegen mangelnder
	Beschlussfähigkeit, nicht mehr behandelt werden konnten, sollen priorisiert
	behandelt werden.
\end{quote}

\subsection*{Begründung}
Diese Änderung fügt auch passiv vertagte Anträge zur Priorisierung für das
nächste Planum hinzu.

\end{document}
