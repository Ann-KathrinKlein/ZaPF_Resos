\documentclass[12pt,oneside]{scrartcl}

% Sprache und Encodings
\usepackage[ngerman]{babel}
\usepackage[T1]{fontenc}
\usepackage[utf8]{inputenc}
\usepackage{epigraph}

% Typographisch interessante Pakete
\usepackage{microtype} % Randkorrektur und andere Anpassungen

% References to Internet and within the document
%\usepackage[pdftex,colorlinks=false,
%pdftitle={Resolution für mehr Lebensqualität},
%pdfauthor={Björn Guth (Aachen), Jörg Behrmann (FUB), Wolfgang Bauer (Alter Sack) und der Rest des Git-Workshops},
%pdfcreator={pdflatex},
%pdfdisplaydoctitle=true]{hyperref}

% Absaetze nicht Einruecken
\setlength{\parindent}{0pt}
\setlength{\parskip}{2pt}

% Formatierung auf A4 anpassen
\usepackage{geometry}
\geometry{paper=a4paper,left=20mm,right=20mm,top=10mm,bottom=10mm}

\renewcommand*\dictumwidth{0.95\linewidth}

\begin{document}

\section*{Resolution zur Netzneutralität in Univeritätsnetzen}

\textbf{Adressaten:} DFN, EDV-Abteilungen deutscher Hochschulen in staatlicher Trägerschaft

\subsection*{Antrag:}
Die ZaPF möge beschließen:
\begin{quote}
Die ZaPF fordert eine uneingeschränkte Einhaltung der Netzneutralität in Universitätsnetzen.

kein trafic shaping

wichtige ports offen
\end{quote}

\subsection*{Begründung:}
\dictum[eduroam.org]{eduroam allows students, researchers and staff from participating institutions to obtain Internet connectivity across campus and when visiting other participating institutions by simply opening their laptop.}

Eine schnelle und stabile Internetverbindung ist heute für ein produktives
Arbeiten unerlässlich. Mit der Einführung von eduroam wurde schon ein großer
Schritt hin zu einer mobilen und vernetzten internationalen
Forschungsgemeinschaft gemacht.

Durch die zahlreichen Besuche bei verschiedensten teilnehmenden Institutionen im
gesamten deutschsprachigen Raum mussten wir jedoch oftmals feststellen, dass
willkürliche und zumeist unverständliche Einschränkungen vielerorts zum Alltag
gehören und das Arbeiten unnötig erschweren.

Zu den bekannten Einschränkungen gehören das Sperren bestimmter Ports oder die
Drosselung langanhalter Verbindungen. Diese und ähnliche Methoden sind
untragbar.

Langanhaltende, schnell Verbindungen sind unerlässlich zur übertragung großer
Datenmengen wie Experimentaldaten und auch wenn wir die Notwendigkeit der
Absicherung einsehen, so muss das oben genannte Mindestmaß an offenen Ports für
eine umstandslose Kommunikation gewährleistet sein.

\vspace{1cm}
\textbf{Verfasser:} Björn Guth (RWTH), Jörg Behrmann (FUB)

\end{document}
