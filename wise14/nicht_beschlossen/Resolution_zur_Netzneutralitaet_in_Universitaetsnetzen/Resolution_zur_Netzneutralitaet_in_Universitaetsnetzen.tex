\documentclass[12pt,oneside]{scrartcl}

% Sprache und Encodings
\usepackage[ngerman]{babel}
\usepackage[T1]{fontenc}
\usepackage[utf8]{inputenc}
\usepackage{booktabs}

% Typographisch interessante Pakete
\usepackage{microtype} % Randkorrektur und andere Anpassungen

% References to Internet and within the document
%\usepackage[pdftex,colorlinks=false,
%pdftitle={Resolution für mehr Lebensqualität},
%pdfauthor={Björn Guth (Aachen), Jörg Behrmann (FUB), Wolfgang Bauer (Alter Sack) und der Rest des Git-Workshops},
%pdfcreator={pdflatex},
%pdfdisplaydoctitle=true]{hyperref}

% Absaetze nicht Einruecken
\setlength{\parindent}{0pt}
\setlength{\parskip}{2pt}

% Formatierung auf A4 anpassen
\usepackage{geometry}
\geometry{paper=a4paper,left=15mm,right=15mm,top=10mm,bottom=10mm}

\renewcommand*\dictumwidth{0.95\linewidth}

\begin{document}

\section*{Resolution zur Netzneutralität in Universitätsnetzen}

\textbf{Adressaten:} Verein zur Förderung eines Deutschen Forschungsnetzwerks e.V., EDV-Abteilungen deutscher Hochschulen in staatlicher Trägerschaft

\subsection*{Antrag:}
Die ZaPF möge beschließen:
\begin{quote}
Die ZaPF fordert eine uneingeschränkte Einhaltung der Netzneutralität in Universitätsnetzen.

Zu den bekannten Einschränkungen gehören das Sperren bestimmter Ports oder die
Drosselung langanhalter Verbindungen. Diese und ähnliche Methoden sind
untragbar.

Langanhaltende, schnell Verbindungen sind unerlässlich zur Übertragung großer
Datenmengen wie Experimentaldaten und auch wenn wir die Notwendigkeit der
Absicherung informationstechnischer Systeme einsehen, so muss das unten genannte Mindestmaß an offenen Ports für
eine umstandslose Kommunikation gewährleistet sein.

\vspace{0.5cm}
\begin{center}
\begin{tabular}{rl}
	\toprule
	Port & Service \\
	\midrule
	22 & SSH \\
	25 & SMTP \\
	80 & HTTP \\
	143 & IMAP \\
	220 & IMAP3 \\
	389 & LDAP \\
	425 & SMTPs \\
	465 & HTTPs \\
	666 & Doom \\
	993 & IMAP via TLS/SSL \\
	6697 & IRC via TLS/SSL \\
	\bottomrule
\end{tabular}
\end{center}
\vspace{0.5cm}
Des weiteren fordern wir die Adressaten auf, eine Stellungnahme zu den durch sie vorgenommenen Maßnahmen abzugeben. Außerdem befürworten wir die Veröffentlichung dieser an geeigneter Stelle im Webauftritt der entsprechenden Hochschule und als Gesamtveröffentlichung durch den Verein zur Förderung eines Deutschen Forschungsnetzes e.V.
\end{quote}

\subsection*{Begründung:}
\dictum[eduroam.org]{eduroam allows students, researchers and staff from participating institutions to obtain Internet connectivity across campus and when visiting other participating institutions by simply opening their laptop.}

Eine schnelle und stabile Internetverbindung ist heute für ein produktives
Arbeiten unerlässlich. Mit der Einführung von eduroam wurde schon ein großer
Schritt hin zu einer mobilen und vernetzten internationalen
Forschungsgemeinschaft gemacht.

Durch die zahlreichen Besuche bei verschiedensten teilnehmenden Institutionen im
gesamten deutschsprachigen Raum mussten wir jedoch oftmals feststellen, dass
willkürliche und zumeist unverständliche Einschränkungen vielerorts zum Alltag
gehören und das Arbeiten unnötig erschweren.


\vspace{1cm}
\textbf{Verfasser:} Jörg Behrmann (FUB), Björn Guth (RWTH)

\end{document}
